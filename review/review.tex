\documentclass[english]{article}

\usepackage{graphicx}
\usepackage{alltt}
\usepackage{url}
%\usepackage{ngerman}
\usepackage{color}
\usepackage{enumitem}


\title{\huge\sffamily\bfseries Review of Group a by Group b}
\author{ w \and x \and y \and z}
\date{\dots}


\begin{document}
\maketitle

%% **** please observe the page limit ****
%% (it is not allowed to change the font size or page geometry to gain more space!)
%% comment or remove lines below before hand-in
\begin{center}
{\large\textcolor{red}{Page limit: 18 pages.}}
\end{center}
%%%%%%%%%%%%%%%%%%%%%%%%%%%%%%%%%%%%%%%%%%%%%%

\tableofcontents
\pagebreak



\section{Background}

\noindent {\bf Developers of the external system:} {\it x', y', z', ...} \\

\noindent {\bf Date of the review:} ...


\section{Report and Design Review}

Study the documentation that came with the external system and evaluation. 

\subsection{System Architecture and Security Concepts}

Is the chosen architecture well-suited for the tasks specified in the requirements? Are the security design decisions well motivated and justified? List positive and negative aspects for each question.


\subsection{Risk Analysis}

Is the risk analysis coherent and complete? That is, are all relevant assets, threat sources, and threats listed? 
%
Are the risk definitions (likelihood, impact, and risk) reasonable?
%
Are the countermeasures appropriate?


\section{Implementation Review}
 
\subsection{Compliance with Requirements} 

%Does the system meet the functional and security requirements given in the assignment? Are they implemented correctly? If not, list any missing functionality %or security measure.
In the following we will analyze the implementation of the functional and security requirements given in the project assignment.

\subsubsection{Functional Requirements}

\begin{enumerate}

\item \textbf{Certificate Issuing}
\item \textbf{Certificate Revocation} 
\item \textbf{CA Administrator Interface}
\item \textbf{Key Backup}
\item \textbf{System Administration and Maintenance}
\end{enumerate}
\subsubsection{Security Requirements}
\begin{enumerate}

\item \textbf{Access Control with regard to CA Functionality and Data}
\item \textbf{Secrecy and Integrity for Private Keys in Backup}
\item \textbf{Secrecy and Integrity for User Data}
\item \textbf{Access Control on all Components}
\begin{enumerate}[label=(\alph*)]
\item \textbf{Firewall}:
\item \textbf{Webbox}:
\item \textbf{Sqlbox}:
\item \textbf{Logbox}:
\end{enumerate}
\end{enumerate}
\subsubsection*{Observations}
input/request: unusual username ( $<$script$>$alert(1)$<$/script$>$)/ request certificate\\
output/response:  certificate request error containing sh exception as html comment $<$ !-- /bin/sh syntax error .... --$>$\\
input/request: shell code as username (Vorname) e.g \$(/bin/ls) then request certificate
output: certificate with evaluated expression in Organizational unit (OU) e.g. app -> backdoor


\subsection{System Security Testing}

Systematically investigate the system. Are the countermeasures implemented as described? Do you see any security problems? Analyze the system using both black-box as well as white-box testing.

\subsubsection{Countermeasures}
In this section we will briefly discuss the implementation of all the countermeasures proposed in the risk analysis.

\paragraph{alskdfj} asdflkjlasdjf



\subsubsection{Title}


\subsubsection{Testing}

Black box testing
\begin{itemize}
	
	\item Nessus scanner
	\item nmap -> 12345 is open UDP
	\item ZAP
	\item By Hand

\end{itemize}

White box testing

\begin{itemize}
	\item Firewall: 12345 -> to webbox\\ openssh-sftp-server installed,  
\end{itemize}



\subsection{Backdoors}

Describe all backdoors that you found on the system. It may be that you also find unintended backdoors (only the group's presentation will show whether they were intended or not).



\section{Comparison}

Compare your system with the external system you were given for the review. Are there any remarkable highlights in your system or the external system?


\end{document}

%%% Local Variables: 
%%% mode: latex
%%% TeX-master: "../../book"
%%% End: 
